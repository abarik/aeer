\usepackage{emoji}
\usepackage{booktabs, caption, longtable, colortbl, array}
\usepackage{hyperref}
\usepackage{biblatex}
\usepackage{icomma}
              
% header.tex
\usepackage[newparttoc] {titlesec}
\usepackage{tocloft}
\usepackage{float}
\usepackage{multirow}
\usepackage{wrapfig}
\usepackage{pdflscape}
\usepackage{tabu}
\usepackage{threeparttable}
\usepackage{threeparttablex}
\usepackage[normalem]{ulem}
\usepackage{makecell}
\usepackage{xcolor}

% Fejezetek (chapter) sorszámozásának módosítása
\titleformat{\chapter}[hang]
  {\normalfont\huge\bfseries}
  {\thechapter.}{1em}{}

% Szintek (section, subsection, stb.) sorszámozásának módosítása
\titleformat{\section}[hang]
  {\normalfont\Large\bfseries}
  {\thesection.}{1em}{}

\titleformat{\subsection}[hang]
  {\normalfont\large\bfseries}
  {\thesubsection.}{1em}{}

\titleformat{\subsubsection}[hang]
  {\normalfont\normalsize\bfseries}
  {\thesubsubsection.}{1em}{}


\renewcommand{\cftchapaftersnum}{.}
\renewcommand{\cftsecaftersnum}{.}
\renewcommand{\cftsubsecaftersnum}{.}
\renewcommand{\cftsubsubsecaftersnum}{.}


% 'Part' szakaszok sorszámozásának engedélyezése római számokkal
\renewcommand{\thepart}{\Roman{part}} % Római számok használata a 'Part' sorszámozásához

% 'Rész' címek formázása a dokumentumban
\titleformat{\part}[display] % 'display' forma külön sorba helyezi a címet
  {\normalfont\Huge\bfseries\filcenter} % Stílus: normál betűkép, nagy méret, félkövér, középre igazítva
  {\thepart. rész} % 'Rész' szó és a sorszám ponttal
  {0pt} % Cím és sorszám közötti távolság
  {\Huge} % A cím mérete
  []

% 'Part' szakaszok megjelenítése a tartalomjegyzékben
\renewcommand{\cftpartaftersnum}{.} % Pont hozzáadása a sorszám után a tartalomjegyzékben
\renewcommand{\cftpartaftersnum}{.} % Pont hozzáadása a sorszám után a tartalomjegyzékben
